\section{Anhang}
Ein Anhang zur wissenschaftlichen Arbeit ist notwendig, wenn Materialien, die die Arbeit als Ganzes oder auch größere Teile derselben betreffen, jedoch nur schwer im Ausführungsteil unterzubringen sind. Das ist insbesondere dann der Fall, wenn sie aufgrund ihres Umfangs den Gesamtzusammenhang der Ausführung stören würden. Inhaltlich darf im Anhang nichts stehen, was zum Verständnis des Textes notwendig ist, der Text der Arbeit darf an dieser Stelle nicht „unter anderen Vorzeichen“ fortgesetzt werden. Er sollte nicht dazu verwendet werden, der Arbeit einen größeren Umfang zu geben und diese „dicker“ erscheinen zu lassen!
Der Anhang eignet sich für ergänzende Dokumente und Materialien, vor allem, falls diese für den Leser nur schwer oder gar nicht zugänglich sind, wie bspw. unveröffentlichte Betriebsunterlagen.
Vor allem in den empirischen Arbeiten kann der Anhang dazu dienen, verwendete Datensätze, eingesetzte mathematisch-statistische Verfahren oder Programme näher zu kennzeichnen. Werden im Rahmen der Untersuchungen Befragungen durchgeführt, sind die Fragestellungen und Ergebnisse im Anhang zu dokumentieren. Auf Gespräche darf im Rahmen der Ausführungen nur dann Bezug genommen werden, wenn ein vom Gesprächspartner unterzeichnetes Ergebnis-Protokoll im Anhang der Arbeit beigefügt ist.
Besteht der Anhang aus mehreren Elementen, so sind die einzelnen Elemente durch Nummerierung voneinander zu trennen.

Hinweis: Bleibt der Anhang leer, kann dieser Abschnitt gelöscht werden
